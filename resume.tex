%%%%%%%%%%%%%%%%%
% This is JD's personal resume
% It is based on altacv.cls (v1.1.3, 30 April 2017) written by LianTze Lim
%%%%%%%%%%%%%%%%

%% If you want to use \orcid or the
%% academicons icons, add "academicons"
%% to the \documentclass options.
%% Then compile with XeLaTeX or LuaLaTeX.
% \documentclass[10pt,a4paper,academicons]{altacv}

%% Use the "normalphoto" option if you want a normal photo instead of cropped to a circle
% \documentclass[10pt,a4paper,normalphoto]{altacv}

\documentclass[10pt,a4paper]{altacv}

%% AltaCV uses the fontawesome and academicon fonts
%% and packages.
%% See texdoc.net/pkg/fontawecome and http://texdoc.net/pkg/academicons for full list of symbols.
%% When using the "academicons" option,
%% Compile with LuaLaTeX for best results. If you
%% want to use XeLaTeX, you may need to install
%% Academicons.ttf in your operating system's font %% folder.


% Change the page layout if you need to
\geometry{left=1cm,right=9cm,marginparwidth=6.8cm,marginparsep=1.2cm,top=1cm,bottom=1cm}

% Change the font if you want to.

% If using pdflatex:
\usepackage[utf8]{inputenc}
\usepackage[T1]{fontenc}
\usepackage[default]{lato}

% If using xelatex or lualatex:
% \setmainfont{Lato}

% Change the colours if you want to
\definecolor{BigRed}{HTML}{E84855}
\definecolor{SlateGrey}{HTML}{2E2E2E}
\definecolor{LightGrey}{HTML}{666666}
\colorlet{heading}{BigRed}
\colorlet{accent}{BigRed}
\colorlet{emphasis}{SlateGrey}
\colorlet{body}{LightGrey}

% Change the bullets for itemize and rating marker
% for \cvskill if you want to
\renewcommand{\itemmarker}{{\small\textbullet}}
\renewcommand{\ratingmarker}{\faCircle}

%% sample.bib contains your publications
\addbibresource{sample.bib}

\begin{document}
\name{John J.D. Davis}
\tagline{Cloud \& Product Security Engineer}
% Cropped to square from https://en.wikipedia.org/wiki/Marissa_Mayer#/media/File:Marissa_Mayer_May_2014_(cropped).jpg, CC-BY 2.0
\photo{2.5cm}{me}
\personalinfo{%
  % Not all of these are required!
  % You can add your own with \printinfo{symbol}{detail}
  \email{me@gmail.com}
  \phone{888-888-8888}
  %\mailaddress{Address, Street, 00000 County}
  \location{San Francisco Bay Area, CA}
%   \orcid{orcid.org/0000-0000-0000-0000} % Obviously making this up too. If you want to use this field (and also other academicons symbols), add "academicons" option to \documentclass{altacv}
}

%% Make the header extend all the way to the right, if you want.
\begin{fullwidth}
\makecvheader
\end{fullwidth}

%% Provide the file name containing the sidebar contents as an optional parameter to \cvsection.
%% You can always just use \marginpar{...} if you do
%% not need to align the top of the contents to any
%% \cvsection title in the "main" bar.
\cvsection[page1sidebar]{Experience}

\cvevent{Lead Security Engineer}{Autodesk}{Oct 2017 -- Ongoing}{San Francisco, CA}
\begin{itemize}
\item Established and led Autodesk's Offensive Security Team, which conducts application \& network penetration tests, architecture reviews, and red team assessments
\item Security hardening for Autodesk's extensive AWS environment
\item Developed Autodesk's threat intelligence program
\end{itemize}

\divider

\cvevent{Senior Security Engineer}{Autodesk}{Jun 2017 -- Oct 2017}{San Francisco, CA}
\begin{itemize}
\item Matured and co-led Autodesk's Cloud Security Incident Response Team (CSIRT); including the development of policies, processes, capabilities, automation, and tools
\item Designed and built multiple in-house security tools for incident response, offensive security, \& various automation projects
\item Worked alongside product and infrastructure teams to remediate vulnerabilities
\end{itemize}

\divider

\cvevent{Application Security Engineer}{Autodesk}{Oct 2015 -- Jun 2017}{San Francisco, CA}
\begin{itemize}
\item Managed continous application scanning across 150+ products
\item Conducted Security Architecture Reviews, Application Penetration Tests, and Vulneraility Assessments
\item Laid the foundation for Autodesk's Offensive Security Team
\end{itemize}

\divider

\cvevent{Security Consultant}{Cigital}{Mar 2015 -- Aug 2015}{Bloomington, IN}

\divider

\cvevent{Lead Security Analyst}{Rook Security}{Feb 2014 -- Mar 2015}{Indianapolis, IN}

\divider

\cvevent{Functionality Tester}{Interactive Intelligence}{Oct 2013 -- Feb 2014}{Indianapolis, IN}

\divider

\cvevent{Technologist}{Indiana University}{Jun 2011 -- Oct 2013}{Indianapolis, IN}

% \cvsection{A Day of My Life}

% % Adapted from @Jake's answer from http://tex.stackexchange.com/a/82729/226
% % \wheelchart{outer radius}{inner radius}{
% % comma-separated list of value/text width/color/detail}
% \wheelchart{1.5cm}{0.5cm}{%
%   10/13em/accent!30/Sleeping \& dreaming about work,
%   25/9em/accent!60/Public resolving issues with Yahoo!\ investors,
%   5/12em/accent!10/New York \& San Francisco Ballet Jawbone board member,
%   20/12em/accent!40/Spending time with family,
%   5/8em/accent!20/Business development for Yahoo!\ after the Verizon acquisition,
%   30/9em/accent/Showing Yahoo!\ employees that their work has meaning,
%   5/8em/accent!20/Baking cupcakes
% }

% \clearpage

% \cvsection[page2sidebar]{Publications}

% \nocite{*}

% \printbibliography[heading=pubtype,title={\printinfo{\faBook}{Books}},type=book]

% \divider

% \printbibliography[heading=pubtype,title={\printinfo{\faFileTextO}{Journal Articles}}, type=article]

% \divider

% \printbibliography[heading=pubtype,title={\printinfo{\faGroup}{Conference Proceedings}},type=inproceedings]

%% If the NEXT page doesn't start with a \cvsection but you'd
%% still like to add a sidebar, then use this command on THIS
%% page to add it. The optional argument lets you pull up the
%% sidebar a bit so that it looks aligned with the top of the
%% main column.
% \addnextpagesidebar[-1ex]{page3sidebar}


\end{document}
